\section{数据结构}

\subsection{树链剖分}

\begin{lstlisting}
/*
  SDOI 2012 染色: 给定一棵树,节点有色。现有两种查询:
  1. 修改某点的颜色
  2. 查询一条链上不同的颜色段数
*/
#include <algorithm>
#include <stdio.h>
#include <string.h>
#include <map>
#define PROB
#define N 100010
#define For(i,n) for(int i=1; i<=n; i++)
using namespace std;

int n;

int le[N],pe[N<<1],ev[N<<1],ecnt;
void addEdge(int u,int v){
  ecnt++;
  pe[ecnt]=le[u];
  le[u]=ecnt;
  ev[ecnt]=v;
}

int ori[N],a[N];

#define lc (o<<1)
#define rc (o<<1|1)
#define mid ((l+r)>>1)
int cl[N<<2],cr[N<<2],sumv[N<<2],setv[N<<2],L,R,setc;
inline void maintain(int o){
  cl[o]=cl[lc];
  cr[o]=cr[rc];
  sumv[o]=sumv[lc]+sumv[rc]-(cr[lc]==cl[rc]);
}
inline void pushdown(int o){
  if(setv[o]!=-1){
    cl[lc]=cr[lc]=setv[lc]=setv[o];
    sumv[lc]=1;
    cl[rc]=cr[rc]=setv[rc]=setv[o];
    sumv[rc]=1;
    setv[o]=-1;
  }
}
void build(int o=1,int l=1,int r=n){
  if(l==r){
    cl[o]=cr[o]=a[l];
    sumv[o]=1;
    return;
  }
  build(lc,l,mid);
  build(rc,mid+1,r);
  maintain(o);
}
void update(int o=1,int l=1,int r=n){
  if(L<=l&&r<=R){
    cl[o]=cr[o]=setv[o]=setc;
    sumv[o]=1;
    return;
  }
  pushdown(o);
  if(L<=mid)update(lc,l,mid);
  if(R>mid)update(rc,mid+1,r);
  maintain(o);
}
struct ANS{
  int cl,cr,sumv;
  ANS(int a,int b,int c):cl(a),cr(b),sumv(c){}
  ANS operator+(const ANS& r){
    return ANS(cl,r.cr,sumv+r.sumv-(cr==r.cl));
  }
};
ANS query(int o=1,int l=1,int r=n){
  if(L<=l&&r<=R)
    return ANS(cl[o],cr[o],sumv[o]);
  pushdown(o);
  if(R<=mid)return query(lc,l,mid);
  if(L>mid)return query(rc,mid+1,r);
  return query(lc,l,mid)+query(rc,mid+1,r);
}

int fa[N],sz[N],hson[N],dep[N];
void dfs1(int u){
  sz[u]=1;
  for(int i=le[u]; i; i=pe[i]){
    int &v=ev[i];
    if(v==fa[u])continue;
    fa[v]=u;
    dep[v]=dep[u]+1;
    dfs1(v);
    sz[u]+=sz[v];
    if(sz[v]>sz[hson[u]])hson[u]=v;
  }
}
int dfs_clock,top[N],id[N];
void dfs2(int u){
  id[u]=++dfs_clock;
  if(hson[u]){
    top[hson[u]]=top[u];
    dfs2(hson[u]);
  }
  for(int i=le[u]; i; i=pe[i]){
    int &v=ev[i];
    if(v==fa[u] || v==hson[u])continue;
    top[v]=v;
    dfs2(v);
  }
}

int Query(int u,int v){
  ANS ans1(-1,-1,0),ans2(-2,-2,0);
  while(top[u]!=top[v]){
    if(dep[top[u]]>dep[top[v]]){
      L=id[top[u]],R=id[u];
      ANS q=query();
      ans1=q+ans1;
      u=fa[top[u]];
    }else{
      L=id[top[v]],R=id[v];
      ANS q=query();
      ans2=q+ans2;
      v=fa[top[v]];
    }
  }
  if(dep[u]>dep[v]){
    L=id[v],R=id[u];
    ANS q=query();
    swap(ans2.cr,ans2.cl);
    return (ans2+q+ans1).sumv;
  }else{
    L=id[u],R=id[v];
    ANS q=query();
    swap(ans1.cl,ans1.cr);
    return (ans1+q+ans2).sumv;
  }
}
void Modify(int u,int v,int c){
  while(top[u]!=top[v]){
    if(dep[top[u]]<dep[top[v]])swap(u,v);
    L=id[top[u]],R=id[u];
    update();
    u=fa[top[u]];
  }
  if(dep[u]<dep[v])swap(u,v);
  L=id[v],R=id[u];
  update();
}
int m;
char op[10];
int main(){
#ifndef ONLINE_JUDGE
  freopen("in.txt","r",stdin);
#endif
  scanf("%d%d",&n,&m);
  For(i,n)scanf("%d",&ori[i]);
  For(i,n-1){
    int u,v;
    scanf("%d%d",&u,&v);
    addEdge(u,v);
    addEdge(v,u);
  }
  dfs1(1);
  top[1]=1;
  dfs2(1);
  For(i,n)a[id[i]]=ori[i];
  memset(setv,-1,sizeof(setv));
  build();
  while(m--){
    int u,v,c;
    scanf("%s",op);
    if(op[0]=='Q'){
      scanf("%d%d",&u,&v);
      printf("%d\n",Query(u,v));
    }else{
      scanf("%d%d%d",&u,&v,&setc);
      Modify(u,v,c);
    }
  }
  return 0;
}

\end{lstlisting}

\subsection{Link-Cut Tree}

\begin{lstlisting}
/*
  水管局局长加强版:加边,维护最小生成树(森林)
*/
#include<stdio.h>
#include<algorithm>
#include<cstring>
#include<vector>
#include<map>
#include<set>
#define REP(i,b,e) for(int i=b; i<=e; i++)
#define RREP(i,b,e) for(int i=b; i>=e; i--)

int getint(){
  char ch = getchar();
  for ( ; ch > '9' || ch < '0'; ch = getchar());
  int tmp = 0;
  for ( ; '0' <= ch && ch <= '9'; ch = getchar())
    tmp = tmp * 10 + int(ch) - 48;
  return tmp;
} 

using namespace std;
#define N 100010
#define M 1000010

int pa[N];
int findset(int u){
  if(pa[u]!=u)pa[u]=findset(pa[u]);
  return pa[u];
}

struct Node{
  int v,rev,maxv;
  int ch[2],fa,pfa;
  inline void pushdown();
  inline void maintain();
  inline void change(int,int);
}node[N+M];
inline void Node::pushdown(){
  if(rev){
    node[ch[0]].rev^=1;
    node[ch[1]].rev^=1;
    swap(ch[0],ch[1]);
    rev=0;
  }
}
inline void Node::maintain(){
  maxv=max(v,max(node[ch[0]].maxv,node[ch[1]].maxv));
}
inline void Node::change(int v,int cur){
  node[v].fa=node[ch[1]].pfa=cur;
  node[v].pfa=node[ch[1]].fa=0;
  ch[1]=v;
  maintain();
}
void rotate(int o,int d){
  int k=node[o].ch[d];
  node[o].ch[d]=node[k].ch[1^d];
  node[node[k].ch[1^d]].fa=o;

  int d2=node[node[o].fa].ch[1]==o;
  node[node[o].fa].ch[d2]=k;
  node[k].fa=node[o].fa;
  
  node[k].ch[1^d]=o;
  node[o].fa=k;
  node[o].maintain();
  node[k].maintain();
}
int top[N+M];
void splay(int cur){
  top[0]=0;
  while(cur){
    top[++top[0]]=cur;
    cur=node[cur].fa;
  }
  cur=top[1];
  if(top[0]>1){
    node[cur].pfa=node[top[top[0]]].pfa;
    node[top[top[0]]].pfa=0;
  }
  RREP(i,top[0],1)
    node[top[i]].pushdown();
  int fa,gfa,d1,d2;
  while(node[cur].fa){
    fa=node[cur].fa,gfa=node[fa].fa;
    d1=node[gfa].ch[1]==fa,d2=node[fa].ch[1]==cur;
    if(!gfa){
      rotate(fa,d2);
    }else{
      if(d1==d2){rotate(gfa,d1);rotate(fa,d2);}
      else{rotate(fa,d2);rotate(gfa,d1);}
    }
  }
  node[cur].pushdown();
}
void access(int u){
  for(int v=0; u; u=node[v].pfa){
    splay(u);
    node[u].change(v,u);
    v=u;
  }
}
void evert(int u){
  access(u);
  splay(u);
  node[u].rev^=1;
}
void cut(int u,int v){
  evert(u);
  access(v);
  splay(u);
  node[u].change(0,u);
  node[v].pfa=0;
}
void link(int u,int v){
  evert(v);
  splay(v);
  access(u);
  splay(u);
  node[v].pfa=u;
}
int getmax(int u,int v){
  evert(u);
  access(v);
  splay(v);
  return node[v].maxv;
}
int findmax(int ret){
  while(node[ret].v!=node[ret].maxv){
    node[ret].pushdown();
    if(node[node[ret].ch[0]].maxv==node[ret].maxv)
      ret=node[ret].ch[0];
    else 
      ret=node[ret].ch[1];
  }
  return ret;
}

int n,m,Q;
struct Query{int k,x,y;}q[N];
struct E{int u,v,w;bool operator<(const E& rhs)const{return w<rhs.w;}}e[M];
map<int,int> id[N];
int del[M],ans[N];
int main(){
#ifdef QWERTIER
  freopen("in","r",stdin);
#endif
  n=getint();m=getint();Q=getint();
  REP(i,1,n)pa[i]=i;
  REP(i,1,m){ e[i].u=getint();e[i].v=getint();e[i].w=getint();}
  sort(e+1,e+m+1);
  REP(i,1,m)id[e[i].u][e[i].v]=id[e[i].v][e[i].u]=i;
  REP(i,1,Q){
    q[i].k=getint();q[i].x=getint();q[i].y=getint();
    if(q[i].k==2)
      del[id[q[i].x][q[i].y]]=1;
  }
  REP(i,1,m){
    node[i+n].v=node[i+n].maxv=e[i].w;
    if(del[i])continue;
    if(findset(e[i].u)!=findset(e[i].v)){
      pa[pa[e[i].u]]=pa[e[i].v];
      link(e[i].u,i+n);
      link(e[i].v,i+n);
    }
  }
  RREP(i,Q,1){
    int &k=q[i].k,&x=q[i].x,&y=q[i].y;
    if(k==1){
      ans[i]=getmax(x,y);
    }else{
      int maxw=getmax(x,y);
      if(maxw<=e[id[x][y]].w)continue;
      else{
        splay(y);
        int u=findmax(y);
        splay(u);
        cut(e[u-n].u,u);
        cut(e[u-n].v,u);
        link(x,n+id[x][y]);
        link(y,n+id[x][y]);
      }
    }
  }
  REP(i,1,Q)if(q[i].k==1)printf("%d\n",ans[i]);
  return 0;
}

\end{lstlisting}
