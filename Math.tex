\section{数学}

\subsection{素数表(NTT专用)}

\begin{table}[htbp]
  \vspace{1cm}
  \centering
  \begin{tabular}{ c | c | c | c }
    \hline
        {\bf $r \cdot 2^k$} & {\bf r} & {\bf k} & {\bf g}\\
        
        \hline

        3	& 1	& 1	& 2 \\
        \hline
        5	& 1	& 2	& 2 \\
        \hline
        17	&	1	& 4	&3 \\
        \hline
        97	&	3	&	5	&5 \\
        \hline
        193	&	3	&	6	&5 \\
        \hline
        257	&	1	&	8	&3 \\
        \hline
        7681	&	15	&	9	&17 \\
        \hline
        12289	&	3	&	12	&11 \\
        \hline
        40961	&	5	&	13	&3 \\
        \hline
        65537	&	1	&	16	&3 \\
        \hline
        786433	&	3	&	18	&10 \\
        \hline
        5767169	&	11	&	19	&3 \\
        \hline
        7340033	&	7	&	20	&3 \\
        \hline
        23068673	&	11	&21	&3 \\
        \hline
        104857601	&	25	&22	&3 \\
        \hline
        167772161	&	5	&25	&3 \\
        \hline
        469762049	&	7	&26	&3 \\
        \hline
        1004535809	&	479	&21	&3 \\
        \hline
        2013265921	&	15	&27	&31 \\
        \hline
        2281701377	&	17	&27	&3 \\
        \hline
        3221225473	&	3	&30	&5 \\
        \hline
        75161927681	&	35	&31	&3 \\
        \hline
        77309411329	&	9	&33	&7 \\
        \hline
  \end{tabular}
\end{table}
\clearpage
\begin{table}[htbp]
  \vspace{1cm}
  \centering
  \begin{tabular}{ c | c | c | c }
    \hline
    {\bf $r \cdot 2^k$} & {\bf r} & {\bf k} & {\bf g}\\
    \hline
    206158430209	&	3	&36	&22 \\
    \hline
    2061584302081	&	15	&37	&7 \\
    \hline
    2748779069441	&	5	&39	&3 \\
    \hline
    1231453023109121	&35	&45	&	3 \\
    \hline
    6597069766657	&	3	&41	&5 \\
    \hline
    39582418599937	&	9	&42	&5 \\
    \hline
    79164837199873	&	9	&43	&5 \\
    \hline
    263882790666241	&	15	&44	&7 \\
    \hline
    1337006139375617	&19	&46	&	3 \\
    \hline
    3799912185593857	&27	&47	&	5 \\
    \hline
    4222124650659841	&15	&48	&	19 \\
    \hline
    7881299347898369	&7	&	50	&	6 \\
    \hline
    31525197391593473	&7	&	52	&	3 \\
    \hline
    180143985094819841	&5	&	55	&	6 \\
    \hline
    1945555039024054273	&27	&	56	&	5 \\
    \hline
    4179340454199820289	&29	&	57	&	3 \\
    \hline

  \end{tabular}
\end{table}

\subsection{NTT}

\begin{lstlisting}
#define MOD 65970697685434369ll
namespace NTT {
int m;
#define eps (1e-15)
LL fMul(LL t, LL p) {
  LL ret = t * p - (LL)((long double)t/MOD*p+eps)*MOD;
  if (ret < 0)
    return (ret % MOD + MOD) % MOD;
  else
    return ret;
}
LL fPow(LL t, LL p) {
  LL ret = 1;
  while (p) {
    if (p & 1)
      ret = fMul(ret, t);
    t = fMul(t, t);
    p >>= 1;
  }
  return ret;
}
int rev[N];
LL g = 19;
void ntt(LL *in, LL *out, int n, int flag) {  // flag = 0 if ntt, 1 if intt
  memset(rev, 0, sizeof(rev));
  REP (i, (1<<m)) {
    REP (j, m) {
      rev[i] |= ((i>>j)&1) << (m-j-1);
    }
  }
  for (int i = 0; i < n; i++) 
    out[rev[i]] = in[i];
  for (int s = 1; (1 << s) <= n; s++) {
    int m = 1 << s, m_2 = m /2;
    LL wm;
    if (!flag)
      wm = fPow(g, (MOD - 1) / m);
    else
      wm = fPow(fPow(g, (MOD - 1) / m), MOD - 2);
    for (int k = 0; k < n; k+=m) {
      LL w = 1;
      for (int j = 0; j < m_2; j++) {
        LL t = fMul(w, out[k + j + m_2]),
            u = out[k + j];
        out[k + j] = (u + t) % MOD;
        out[k + j + m_2] = (u - t + MOD) % MOD;
        w = fMul(w, wm);
      }
    }
  }
  if (flag) {
    LL inv = fPow(n, MOD - 2);
    REP (i, n) {
      out[i] = fMul(out[i], inv);
    }
  }
}
}
\end{lstlisting}

\subsection{FFT}

\begin{lstlisting}
namespace FFT {
int L;
struct cp {
  double r,i;
  cp(){r=i=0;}
  cp(double _r,double _i):r(_r),i(_i){}
  cp operator*(cp &t){return cp(r*t.r-i*t.i,i*t.r+t.i*r);}
  cp operator+(cp &t){return cp(r+t.r,i+t.i);}
  cp operator-(cp &t){return cp(r-t.r,i-t.i);}
}; 
const double PI=3.1415926535897932384626;
cp tmp[N];
int rev[N];
void fft(cp *a,int flag) {
  REP(i,(1<<L))tmp[i]=a[rev[i]];
  REP(i,(1<<L))a[i]=tmp[i];
  for(int k=2; k<=(1<<L); k<<=1){
    cp wn(cos(2*PI/k),flag*sin(2*PI/k));
    for(int i=0; i<(1<<L); i+=k){
      cp w(1,0),x,y;
      for(int j=i; j<i+k/2; j++){
        x=a[j];
        y=a[j+k/2]*w;
        a[j]=x+y;
        a[j+k/2]=x-y;
        w=w*wn;
      }
    }
  }
  if(flag==-1)REP(i,(1<<L))a[i].r/=(1<<L);
}
void calc_rev() {
  REP(i,(1<<L)){
    rev[i] = 0;
    REP(j,L) {
      if((i>>j)&1)
        rev[i]|=(1<<(L-1-j));
    }
  }
}
}
\end{lstlisting}

\subsection{FWT}
\begin{lstlisting}
namespace FWT {
template <typename T>
void or_fwt(T X[], int l, int r) {
  if (l == r)
    return;
  int m = (l + r) >> 1;
  or_fwt(X, l, m); or_fwt(X, m + 1, r);
  for (int i = 0; i <=  m - l; i++) {
    X[m + 1 + i] += X[l + i];
  }
}
template <typename T>
void or_ifwt(T X[], int l, int r) {
  if (l == r)
    return;
  int m = (l + r) >> 1;
  or_ifwt(X, l, m); or_ifwt(X, m + 1, r);
  for (int i = 0; i <=  m - l; i++) {
    X[m + 1 + i] -= X[l + i];
  }
}
template <typename T>
void and_fwt(T X[], int l, int r) {
  if (l == r)
    return;
  int m = (l + r) >> 1;
  and_fwt(X, l, m); and_fwt(X, m + 1, r);
  for (int i = 0; i <=  m - l; i++) {
    X[l + i] += X[m + 1 + i];
  }
}
template <typename T>
void and_ifwt(T X[], int l, int r) {
  if (l == r)
    return;
  int m = (l + r) >> 1;
  and_ifwt(X, l, m); and_ifwt(X, m + 1, r);
  for (int i = 0; i <=  m - l; i++) {
    X[l + i] -= X[m + 1 + i];
  }
}
template <typename T>
void xor_fwt(T X[], int l, int r) {
  if (l == r)
    return;
  int m = (l + r) >> 1;
  xor_fwt(X, l, m); xor_fwt(X, m + 1, r);
  for (int i = 0; i <=  m - l; i++) {
    X[l + i] += X[m + 1 + i];
    X[m + 1 + i] = X[l + i] - 2 * X[m + 1 + i];
  }
}
template <typename T>
void xor_ifwt(T X[], int l, int r) {
  if (l == r)
    return;
  int m = (l + r) >> 1;
  for (int i = 0; i <=  m - l; i++) {
    X[l + i] += X[m + 1 + i];
    X[m + 1 + i] = X[l + i] - 2 * X[m + 1 + i];
    X[l+i] /= 2;
    X[m + 1 + i] /= 2;
  }
  xor_ifwt(X, l, m); xor_ifwt(X, m + 1, r);
}
}
\end{lstlisting}

\subsection{拓展欧几里得}
\begin{lstlisting}
LL exgcd(LL a, LL b, LL &x, LL &y) {
  if (a==0&&b==0)return -1;
  if (b==0) {x=1;y=0;return a;}
  LL d=extend_gcd(b,a%b,y,x);
  y-=a/b*x;
  return d;
}
\end{lstlisting}


\subsection{米勒罗宾素数测试}

\begin{lstlisting}
const int S = 8;
long long mult_mod(long long a,long long b,long long c) {
  a %= c;
  b %= c;
  long long ret = 0;
  long long tmp = a;
  while(b) {
    if(b & 1) {
      ret += tmp;
      if(ret > c)ret -= c;
    }
    tmp <<= 1;
    if(tmp > c)tmp -= c;
    b >>= 1;
  }
  return ret;
}
long long pow_mod(long long a,long long n,long long mod) {
  long long ret = 1;
  long long temp = a%mod;
  while(n) {
    if(n & 1)ret = mult_mod(ret,temp,mod);
    temp = mult_mod(temp,temp,mod);
    n >>= 1;
  }
  return ret;
}
bool check(long long a,long long n,long long x,long long t) {
  long long ret = pow_mod(a,x,n);
  long long last = ret;
  for(int i = 1;i <= t;i++) {
    ret = mult_mod(ret,ret,n);
    if(ret == 1 && last != 1 && last != n-1)return true;//合数
    last = ret;
  }
  if(ret != 1)return true;
  else return false;
}
bool Miller_Rabin(long long n) {
  if( n < 2)return false;
  if( n == 2)return true;
  if( (n&1) == 0)return false;//偶数
  long long x = n - 1;
  long long t = 0;
  while( (x&1)==0 ){x >>= 1; t++;}
  srand(time(NULL));
  for(int i = 0;i < S;i++) {
    long long a = rand()%(n-1) + 1;
    if( check(a,n,x,t) )
      return false;
  }
  return true;
}
\end{lstlisting}

\subsection{质因数分解}

\begin{lstlisting}
long long factor[100];//质因素分解结果(刚返回时时无序的)
int tol;//质因数的个数,编号0~tol-1
long long gcd(long long a,long long b) {
  long long t;
  while(b) {
    t = a;
    a = b;
    b = t%b;
  }
  if(a >= 0)return a;
  else return -a;
}
// 找出一个因子
long long pollard_rho(long long x,long long c) {
  long long i = 1, k = 2;
  srand(time(NULL));
  long long x0 = rand()%(x-1) + 1;
  long long y = x0;
  while(1) {
    i ++;
    x0 = (mult_mod(x0,x0,x) + c)%x;
    long long d = gcd(y - x0,x);
    if( d != 1 && d != x)return d;
    if(y == x0)return x;
    if(i == k){y = x0; k += k;}
  }
}
//对 n进行素因子分解,存入factor. k设置为107左右即可
void findfac(long long n,int k) {
  if(n == 1)return;
  if(Miller_Rabin(n)) {
    factor[tol++] = n;
    return;
  }
  long long p = n;
  int c = k;
  while( p >= n)
    p = pollard_rho(p,c--);//值变化,防止死循环k
  findfac(p,k);
  findfac(n/p,k);
}
\end{lstlisting}

\subsection{高斯消元}
\begin{lstlisting}
bool gauss(double *m[N], int n) {
  for(int i=0; i<n; i++){
    int c=i;
    for(int j=i; j<n; j++)if(fabs(m[j][i])>fabs(m[c][i]))c=j;
    if (fabs(m[c][i]) < eps)
      return false;
    for(int j=0; j<=n; j++)swap(m[i][j],m[c][j]);
    for(int j=i+1; j<n; j++){
      double f=m[j][i]/m[i][i];
      for(int k=n; k>=i; k--)
        m[j][k]-=m[j][i]*m[i][k]/m[i][i];
    }
  }
  for(int i=n-1; i>=0; i--){
    for(int j=i+1; j<n; j++)
      m[i][n]-=m[j][n]*m[i][j];
    m[i][n]/=m[i][i];
  }
  return true;
}
\end{lstlisting}
